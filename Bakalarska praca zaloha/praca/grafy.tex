\chapter{Grafy}
\label{kap:grafy}

V tejto kapitole uvedieme zopár definícií týkajúcich sa grafov, opíšeme niektoré grafové štruktúry, prípadne zavedieme rôzne názvy, ktoré budeme v ďalších častiach práce potrebovať.


\section{Definícia grafu}

Začneme so základnou definícou - s definovaním pojmu \textit{graf}. Tú prevezmeme od Reinharda Diestela \cite[kapitola 0.1]{Diestel2000}: \newline

\textit{Graf} je dvojica $G = (V, E)$ disjunktných množín, kde prvky $E$ sú dvojprvkové podmnožiny $V$. Prvky $V$ sú \textit{vrcholy} (prípadne \textit{uzly} alebo \textit{body}) grafu $G$, prvky $E$ sú jeho hrany. \newline

Vytvorený objekt by sa mal korektne nazývať \textit{neorientovaný graf}. Avšak my, ako aj mnohí iní, od tohto pomenovania upustíme a budeme pre jednoduchosť používať len pojem \textit{graf}.\newline

Hranu $\{x, y\}$, kde $x,y\in V$, budeme zasa zvyčajne označovať aj ako $(x, y)$ alebo $xy$.


\section{Základné pojmy týkajúce sa grafov}

V nasledujúcej podkapitole sú uvedené definície rôznych pojmov, ktoré charakterizujú a popisujú vlastnosti grafových štruktúr.\newline

\textit{Rád grafu} definujeme ako počet vrcholov grafu $G$.\newline

Budeme hovoriť, že vrchol $v$ je \textit{incidentný} s hranou $e$, ak $v \in e$. Čiže, ak hrana $e$ = $xy$, oba vrcholy $x$ aj $y$ nazveme incidentnými s hranou $e$. Ak sú dva rôzne vrcholy incidentné s tou istou hranou, budeme ich volať jej \textit{koncovými vrcholmi}.\newline

Dva vrcholy $x, y \in V$ sú  \textit{susedné}, ak $xy$ je hrana v $G$.
Dve hrany $e, f \in E; e \neq f$ sú  \textit{susedné}, ak majú spoločný vrchol.\newline

\textit{Stupňom vrchola} $v \in V$ nazveme počet hrán incidentných s $v$. Alebo inak povedané, je to počet vrcholov, ktoré sú susedné s vrcholom $v$.


\subsection{Typy grafov}

Teraz uvedieme niektoré základné typy grafov.\newline

\textit{Prázdny graf} je graf $G = (\emptyset, \emptyset)$. Označenie zjednodušíme na $G = \emptyset$ .\newline

Pojmom \textit{Triviálny graf} budeme označovať graf s rádom 0 alebo 1.\newline

\textit{Kompletným grafom} nazveme taký graf, ktorého všetky vrcholy sú navzájom susedné. Uvažujme $n$ vrcholov, potom kompletný graf na týchto vrcholoch označíme $K_{n}$. Na predstavu môže poslúžiť príklad kompletného grafu troch vrcholov $K_{3}$, čo je trojuholník.\newline

Majme graf $G$. Ak každý jeho vrchol má rovnaký stupeň $k$, potom tento graf nazveme \textit{k-regulárnym}. V definícií môžeme upustiť od počtu vrcholov a nazvať vzniknutý objekt iba \textit{regulárnym}.\newline


\subsection{Vzťahy medzi grafmi}

Nech $G = (V, E)$ a $G' = (V', E')$ sú grafy. Označíme $G\cap G' = (V\cap V', E\cap E')$ a $G\cup G' = (V\cup V', E\cup E')$. Ak prienik týchto dvoch grafov je prázdny graf $G\cap G' = \emptyset$, potom $G$ a $G'$ sú \textit{disjunktné}. Ak $V'\subseteq V$ a $E'\subseteq E$, potom hovoríme, že $G'$ je podgraf $G$ (alebo $G$ je nadgraf $G'$, či $G$ obsahuje $G'$) a píšeme $G'\subseteq G$.\newline


\section{Cesty a cykly}

Nech $G = (V, E)$ je graf. Definujme \textit{cestu} ako neprázdnu striedavú postupnosť $v_{0}e_{0}v_{1}e_{1} ... e_{k-1}v_{k}$, kde $ v_{i} \in V, e_{j} \in E$ pre $i = 1, 2, ..., k$ a $j = 1, 2, ..., k-1$, pričom $e_{i} = v_{i}v_{i+1}$ pre všetky $i < k$ a všetky $v_{i}$ sú navzájom rôzne. Neformálne povedané, cesta spája (pomocou hrán) dva vrcholy grafu $G$, pričom sa v nej použité vrcholy nesmú opakovať. Počet hrán cesty sa nazýva aj \textit{dĺžkou}.\newline

Majme graf $G = (V, E)$. Nech $P = v_{0}e_{0}v_{1}e_{1} ... e_{k-1}v_{k}$ je cesta v $G$, pričom $k\geq 2$ a v grafe $G$ existuje hrana $v_{0}v_{k}$. Potom hovoríme, že graf $G$ obsahuje \textit{cyklus} alebo tiež \textit{kružnicu} $C$. Cyklus definujeme opäť ako neprázdnu striedavú postupnosť vrcholov a hrán, v tomto prípade  $C = $v_{0}e_{0}v_{1}e_{1} ... e_{k-1}v_{k}e_{k}$, kde $e_{k} = \{v_{0},v_{k}\}$.\newline

\textit{Sled} v grafe $G$ je názov pre neprázdnu striedavú postupnosť $v_{0}e_{0}v_{1}e_{1} ... e_{k-1}v_{k}$ vrcholov a hrán v $G$, pričom $e_{i} = v_{i}v_{i+1}$ pre všetky $i < k$. Sled teda, na rozdiel od cesty, môže obsahovať rovnaké vrcholy viac krát. Preto sa dá povedať, že sled je cesta, v ktorej sa môžu vrcholy opakovať. Ale aj naopak: sled, v ktorom sú vrcholy navzájom rôzne, je cestou.\newline

Sled, v ktorom sú hrany navzájom rôzne, sa nazýva \textit{ťah}.\newline


\section{Súvislosť}

Ak pre ľubovoľné dva vrcholy grafu $G$ existuje cesta, tento graf nazveme \textit{súvislým}. \newline

\textit{Komponent grafu} $G$ je súvislý podgraf grafu $G$ taký, že nie je obsiahnutý v žiadnom väčšom súvislom podgrafe grafu $G$. Napríklad, súvislý graf má práve jeden komponent, a to samého seba.\newline

Hrana $e \in E$ grafu $G$ sa volá \textit{most}, ak graf $G$ má menší počet komponentov v porovnaní s grafom $G$ bez hrany $e$. Čiže, po odstránení hrany z grafu pribudne práve jeden komponent.\newline

Vrchol $v \in V$ grafu $G$ sa nazýva \textit{artikulácia}, ak počet komponentov grafu $G$ je menší ako ich počet po odstránení $v$ z grafu $G$. Teda, ak po odstránení vrchola z grafu pribudne aspoň jeden komponent.\newline


\section{Kostry, stromy a lesy}

Nech $G = (V, E)$ je súvislý graf. \textit{Kostrou} grafu nazveme taký graf $G' = (V', E'), G'\subseteq G$, pre ktorý platí $V' = V$ a navyše medzi každými dvoma vrcholmi existuje práve jedna cesta. Ak by sme grafu $G$ postupne odnímali hrany tak, aby sme nenarušili jeho súvislosť, tak po odstránení všetkých takýchto hrán by sme získali kostru grafu $G$. Z tejto konštrukčnej definície vyplýva, že všetky hrany v kostre grafu sú mostami. Je z nej taktiež zrejmé, že každá kostra grafu je súvislá.\newline

\textit{Kostrový les} je taký graf $G$, pre ktorý platí, že každý z jeho komponentov je kostrou.\newline

Acyklický graf (to jest taký, ktorý neobsahuje cyklus), ktorý je navyše súvislý, nazveme \textit{stromom}. Všetky vrcholy stromu so stupňom 1 sú jeho \textit{listy}. Je vhodné si povšimnúť, že, ako pri kostre, všetky hrany sú mostami. Taktiež je zaujímavé uvažovať o rozdiele medzi stromom a kostrou grafu. Možno nahliadnuť, že jediným rozdielom medzi nimi je, že strom je grafom sám o sebe, zatiaľ čo o kostre hovoríme len v súvislosti v grafom, z ktorého vznikla.\newline

\textit{Les} je acyklický graf. Takže každý jeho komponent je stromom.\newline


\section{Špeciálne grafové štruktúry}

Graf, ktorého prvky sú ohodnotené číslom určujúcim cenu, prípadne výhodnosť prechodu cezeň, nazveme \textit{ohodnotený graf}. \textit{Hranovo ohodnotený graf} je graf s funkciou $w: E\rightarrow R$. Teda pre ľubovoľnú hranu $e$ existuje číslo $w(e)$, ktoré nazveme \textit{hodnotou hrany}, prípadne \textit{cenou hrany}.  Z praktického hľadiska je výhodné zadefinovať aj \textit{kladne hranovo ohodnotený graf}. A to je taký, že $w(e) > 0$ pre všetky hrany. \textit{Vrcholovo ohodnotený graf} je graf s funkciou $w: V\rightarrow R$. Teda pre ľubovoľný vrchol $v$ existuje číslo $w(v)$. \newline

\textit{Orientovaný graf} je dvojica $G = (V, E)$, kde každý prvok $xy \in E$ vyjadruje usporiadanú dvojicu. Neformálne, ak je daná hrana $xy$, ale nie $yx$, existuje cesta dĺžky 1 z vrcholu $x$ do $y$, ale z vrcholu $y$ do $x$ takáto cesta nejestvuje.\newline

Rozdiel medzi orientovanými a neorientovanými grafmi je, ako názov napovedá v existencií orientácií hrán. Ak by sme túto skutočnosť chceli vyjadriť formálne, povedali by sme, že zatiaľ čo v neorientovanom grafe obsahuje množina $E$ dvojprvkové množiny, v orientovanom sa skladá z usporiadaných dvojíc.\newline


\subsection{Cesty a cykly pre orientované grafy}

Definícia cesty i cyklu pre orientované grafy je rovnaká ako pre neorientované. Rozdiel je len v jej použití na orientovaných grafoch. Pre úplnosť ich ale v stručnej podobe uvádzame.\newline

Nech $G = (V, E)$ je orientovaný graf. Nech $P = $v_{0}e_{0}v_{1}e_{1} ... e_{k-1}v_{k}$, kde $ v_{i} \in V, e_{j} \in E$ pre $i = 1, 2, ..., k$ a $j = 1, 2, ..., k-1$, pričom $e_{i} = v_{i}v_{i+1}$ pre všetky $i < k$ a  všetky $x_{i}$ sú po dvoch rôzne. Potom $P$ nazývame \textit{cestou} v grafe $G$. Ak $k\geq 3$ a vrcholy sú rôzne až na rovnosť $x_{0} = x_{k}$, potom nazveme $P$ \textit{cyklom}, prípadne \textit{kružnicou}. \cite[kapitola 1.4]{bang2008digraphs}



