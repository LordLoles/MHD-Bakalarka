\chapter*{Úvod} % chapter* je necislovana kapitola
\addcontentsline{toc}{chapter}{Úvod} % rucne pridanie do obsahu
\markboth{Úvod}{Úvod} % vyriesenie hlaviciek

Cieľom tejto práce je spracovať informácie a vytoriť prehľad o algoritmoch, ktoré sú použiteľné  pri vyhľadávaní spojenia v grafikone hromadnej dopravy a následne stručne popísať ich výhody i nedostatky pri ich využití na takéto vyhľadávanie. Praktickou stránkou práce je implementácia algoritmu, ktorý sme zhodnotili, že je najvhodnejší na vyhľadávanie a zakomponovanie algoritmu do aplikácie vhodnej na používanie. Ďalším krokom je rozšírenie programu o dodatočnú funkcionalitu, konkrétne vizualizáciu alternatívnych trás.\newline

V kapitole \ref{kap:grafy} oboznámime čitateľa so základnými pojmami, ako je napríklad graf či cesta, ako i s ďalšími potrebnými definíciami, ktoré budú využívané v ostatnych kapitolách.\newline

V kapitole \ref{kap:algoritmy} popíšeme algoritmy aplikovateľné na grafové štruktúry a využiteľné pri vyhľadávaní spojení hromadnej dopravy. Pre každý algoritmus uvážime, prečo je vhodný pre takéto vyhľadávanie a prečo nie.\newline

V kapitole \ref{kap:softver} sa pozrieme na softvér, ktorý sme implementovali. Uvedieme jeho možnosti a funkcie, popíšeme, ako ho ovládať a ukážeme i nejaké príklady z jeho použitia.\newline

V kapitole \ref{kap:implementacia} odhalíme, aké úvahy a myšlienky nás doprevádzali, aké postupy, algoritmy či štruktúry sme zvolili pri implementácií opísaného softvéru.\newline