\chapter*{Záver}  % chapter* je necislovana kapitola
\addcontentsline{toc}{chapter}{Záver} % rucne pridanie do obsahu
\markboth{Záver}{Záver} % vyriesenie hlaviciek

V práci sme skúmali vlastnosti a popisovali algoritmy vhodné na vyhľadávanie spojení hromadnej dopravy. Zamerali sme sa na tri základné algoritmy, ktoré sme medzi sebou v mnohých ohľadoch porovnávali, vďaka čomu sme dokázali usúdiť, ktorý z nich bude najvhodnejším kandidátom na spomenuté vyhľadávanie. Porovnávali sme ich časovú zložitosť, ich použiteľnosť na malých i veľkých vstupoch, resp. na vstupoch s malým či veľkým počtom hrán. Pre účely našej práce obstál najlepšie Dijkstrov algoritmus, ktorý sa zdá byť najvšeobecnejší a najefektívnejší pri použití ľubovoľnej grafovej štruktúry. Samozrejme, ako sa spomína aj v \cite{bast2010fast}, existujú rôzne zložitejšie algoritmy, ktoré sú oveľa rýchlejšie na vyhľadávanie spojení MHD. Mnoho výskumov a návrhov komplikovanejších algoritmov zhrnuli a porovnali Fan a Machemehl v svojom diele \cite{fan2004optimal}.

So znalosťou Dijkstrovho algoritmu sme dokázali vytvoriť softvér, ktorý na poskytnutých dátach nájde najlacnejšie cesty medzi zadanými vrcholmi. Program si najprv z dát vytvorí grafovú štruktúru, s ktorou následne pracuje. Program je navrhnutý tak, že vďaka vstupu od požívateľa vo forme názvu počiatočnej a koncovej zastávky dokáže uskutočniť vyhľadávanie a následne vypísať vypočítané výsledky v prívetivej forme. Aplikáciu sme otestovali i na dátach Bratislavskej hromadnej dopravy, pričom nás prekvapila rýchlosť algoritmu, ktorému i pri tak obrovskom počte zastávok a liniek trvalo nájsť riešenie zakaždým menej ako sekundu. Samozrejme tomu prispieva prítomnosť ukončenia vyhľadávania, keď už riešenie obsahuje dostatočne veľa potrebných informácií.

Navrhli sme a implementovali zobrazovanie alternatívnych trás do danej zastávky. Takéto vylepšenie zväčša iné implementácie podobného charakteru neposkytujú, a preto sme sa preň rozhodli.

Na um nám prichádzajú i možné pokračovania práce. Napríklad by sa dalo spracovať i iné, pokročilejšie algoritmy a na základe ich spoločného porovnania vybrať lepší algoritmus ako Dijkstrov na vyhľadávanie spojení hromadnej dopravy. Ďalšou možnosťou je implementovať viacero algoritmov na vyhľadávanie a tie na základe rozumných meraní medzi sebou porovnávať. Taktiež by sa dalo implementovať iné použiteľné vylepšenia a zakomponovať ich do kódu našej aplikácie, prípadne zlepšiť či zefektívniť terajšie riešenie, napríklad sprehľadnením či vizuálnym vylepšením používateľského rozhrania.

